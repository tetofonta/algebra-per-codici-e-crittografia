\subsection{Considerazioni}
\begin{frame}
    \frametitle{Considerazioni}
    MIFARE Classic non rappresenta una piattaforma sicura per lo sviluppo di applicazioni contactless, e in
    particolare è inadatto ad applicazioni relative a micropagamenti\pause

    Durante la progettazione della tecnologia sono stati commessi gravi errori che, data la banalità di alcuni,
    potrebbero essere stati inseriti con scopi malevoli~\cite{Courtois2009TheDS}
\end{frame}

\begin{frame}
    \frametitle{Miglioramenti - RNG}
    La maggioranza degli attacchi utilizza la predicibilità del RNG. \pause
    
    La soluzione in questo caso è di utilizzare un TRNG disponibile in hardware, a costo di spazio su silicio e di costi pecuniari maggiorati.\pause

    Un migliormento parziale potrebbe avvenire utilizzando un LFSR da 32 bit e non 16, aumentando così le possibili combinazioni al fine di rallentare gli attacchi.
\end{frame}

\begin{frame}
    \frametitle{Miglioramenti - LFSR}
    Alcune vulnerabilità sono causate dalla funzione di filtraggio del LFSR (Slide~\ref{sec:filter-fn}).
    A tal fine potrebbe essere vantaggiosa un'implementazione dove il cifrario venga sostituito da un modello crittograficamente sicuro.
    
    Una valida proposta potrebbe essere GRAIN-128, cifrario ideato sostanzialmente per ambienti e dispositivi a basso costo e bassissima area occupata\cite{hell2006stream}\pause
\end{frame}
\begin{frame}
    \frametitle{Miglioramenti - LFSR}
    Restano però alcune problematiche:
    \begin{itemize}
        \item <1-> Grain necessita di una chiave di 128bit\newline
                A tal fine è possibile caricare un vaore randomico a seguire della chiave nel cifrario per garantire più entropia dei processi di autenticazione.
        \item <2-> Alternativamente sarbbe necessario aumentare la lunghezza della chiave. Per fare ciò sarebbe poi necessario modificare la struttura di memoria oppure ridurre lo spazio consentito ai dati del tag.
    \end{itemize}
\end{frame}
\note{
    In ogni caso la lunghezza della chiave di 48 bit è da considerarsi non siura e la tecnologia come tale.
}

\begin{frame}
    \frametitle{Miglioramenti - ALGORITMO}
    Una soluzione più drastica è rappresentata dal cambio del circuito di cifratura, passando a un'implementazione AES a basso costo.\cite{feldhofer2005aes}

    In questo caso resta la problematica della gestione di chiavi a 128 bit.
\end{frame}
\subsection{Storia}

\begin{frame}
    \frametitle{Storia}
    \textbf{CRYPTO1} è uno stream cipher sviluppato da \textit{NXP~Semiconductors} nel 1994
    insieme alla famiglia di tag MIFARE Classic~\cite{tezcan2017brute}
\end{frame}


\begin{frame}
    \frametitle{Security through obscurity}
    La sicurezza del sistema era affidata al concetto di \textit{Security through obscurity}

    \begin{itemize}
        \item <2-> Metodo fortemente sconsigliato da tutti gli organi normativi sulla sicurezza\cite{scarfone2008guide}
        \item <3-> Tecnica in contrasto con \textit{Security by Design} e \textit{Open security}
    \end{itemize}
\end{frame}
\note{
    Contrariamente alle due politiche \textit{Security by Design} e \textit{Open security}
    la sicurezza tramite offuscamento è fortemente sconsigliata, in quanto affida la sicurezza del sistema
    al fatto che nessuno riesca a comprenderlo.

    Questa pratica rende quindi il sistema vulnerabile a qualsiasi attacco di tipo reverse engineering,
    oltre che a possibili fughe di informazioni.

    L'utilizzo di ideologie ``open'' permette la validazione del sistema da parte di un maggior numero di enti
    e di membri di una comunità, favorendo così l'individuazione di falle in minor tempo.

    Il metodo più efficiente, però, consiste sempre nell'utilizzo di sistemi già esistenti e ritenuti sicuri (p.e. tritium)
}

\begin{frame}
    \frametitle{La caduta di CRYPTO1}
    Nel 2008/2009 più ricercatori in contemporanea hanno trovato vulnerabilità e sono stati rilasciati attacchi sul critto-sistema CRYPTO1
    che ne hanno interamente distrutto la sicurezza.\cite{garcia2008dismantling}\cite{courtois2008algebraic}\cite{nohl2008reverse}\pause

    Il sistema presenta falle nella sicurezza in più settori:

    \begin{itemize}
        \item <2-> Random Number Generator
        \item <3-> Proprietà algebriche e vulnerabilità strutturali del LFSR
        \item <4-> Complessità delle chiavi (Che rendono il critto-sistema vulnerabile ad attacchi di tipo bruteforce~\cite{courtois2008algebraic})
        \item <5-> Logica di gestione della comunicazione
    \end{itemize}

    (Praticamente presenta falle in ogni sua componente)
\end{frame}

